\chapter{Abstract}
%\addcontentsline{toc}{chapter}{Abstract (Zusammenfassung)}

\begin{center}
	\begin{tabular}{p{3cm}p{10cm}}
		\textbf{Thema:} & \thema \\
		 & \\
		\textbf{Verfasser:} & \autor \\
		 & \\
		\textbf{Betreuer:} & \prueferATitle\ \prueferA \\
		                   & \prueferBTitle\ \prueferB \\
		 & \\
		\textbf{Abgabedatum:} & \abgabedatum \\
		 & \\
		\textbf{Schlagworte:} & Modellierung von Testdaten, Beziehungen in Testdaten, Generieren von Testdaten, DSL f�r Testdaten, Datenbanktest \\
		 & \\
	\end{tabular}
\end{center}

Softwaretests haben sich als Teil der Qualit�tssicherung von Softwareprojekten etabliert.
F�r Tester ist die Modellierung von Testdaten f�r Datenbank-basierte Anwendungen allerdings
nicht immer einfach. Die Daten k�nnen aufgrund von Beziehungen von Datens�tzen schnell
un�bersichtlich und komplex werden. Wegen der Komplexit�t versuchen Tester, mehrere Tests
mit denselben Daten durchzuf�hren.

In dieser Masterarbeit werden eine neue Modellierungssprache f�r Testdaten f�r
Datenbank-basierte Anwendungen und ein Algorithmus zur Generierung von Testdaten 
vorgestellt. Die Sprache erlaubt eine �bersichtliche Beschreibung von Daten und
von Beziehungen zwischen Datens�tzen. Der Algorithmus zur Generierung erzeugt
Daten anhand der Beziehungstypen im Datenbank-Modell. Der Algorithmus versucht
viele Grenzf�lle zu erzeugen, so dass die Daten in m�glichst vielen Tests verwendet
werden k�nnen.

Die Sprache und der Algorithmus wurden implementiert und in praktischen Anwendungen
auf ihre Nutzbarkeit �berpr�ft. Die Implementierung steht unter der Open-Source-Lizenz
\textit{Apache License 2.0}.






\todo{Abstract schreiben}

\todo{Grafik SB Testing DB tauschen}

\todo{auf externe Daten eingehen}
