\chapter{Einleitung}

Softwaretests sind ein Grundpfeiler f�r die Qualit�tssicherung von Anwendungen. 
Um Datenbank-basierte Anwendungen zu testen, m�ssen Test-Daten spezifiziert werden. Man
spricht hier auch von der Modellierung von Test-Daten.

Eine Herausforderung bei der Modellierung stellt nicht nur das Spezifizieren der Daten,
sondern auch das Ausdr�cken von Beziehungen von Datens�tzen dar. Besonders bei Systemen mit 
komplexen Tabellen-Schemata kann die Modellierung un�bersichtlich werden.

F�r den Tester ist eine un�bersichtliche Modellierung aus verschiedenen Gr�nden ein Problem.
Einerseits macht sie die Modellierung fehleranf�lliger. Andererseits ist es schwerer,
die modellierten Daten zu erfassen und zu verstehen. Oftmals werden deshalb f�r verschiedene
Tests dieselben Daten verwendet.

 
- Generierung von Daten mit Fokus auf Beziehungen



\todo{Einleitung schreiben}
- Motivation / allgemeine Problemstellung

- Aufgabenstelung / Zielsetzung


\section{Zielsetzung}

\section{Aufbau der Arbeit}


\todo{Weitere Quellen und Zitate einbauen}

\section*{Ungenutzte Quellen}
\begin{enumerate}
	\item \cite[20ff]{DER_INTEGRATIONSTEST}
	\item \cite{MODELLGETRIEBENE_SOFTWAREENTWICKLUNG}
	\item \cite{DOMAIN_DRIVEN_DESIGN}
\end{enumerate}
