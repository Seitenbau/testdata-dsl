\chapter{Einleitung}

Softwaretests sind ein Grundpfeiler f�r die Qualit�tssicherung von Anwendungen. F�r die
Tests werden Testdaten spezifiziert, anhand derer das Verhalten der Software gepr�ft
wird. Vor allem bei Datenbank-basierten Anwendungen k�nnen solche Testdaten sehr
umfangreich und komplex werden. In diesem Umfeld spricht man auch von der
Modellierung von Testdaten.

Der Grund f�r diese Komplexit�t liegt darin, dass nicht nur Daten beschrieben,
sondern auch Beziehungen von Datens�tzen ausgedr�ckt werden m�ssen. Besonders bei Systemen mit 
komplexen Tabellen-Schemata kann die Modellierung un�bersichtlich werden.

F�r den Tester ist eine un�bersichtliche Modellierung aus verschiedenen Gr�nden ein Problem.
Einerseits macht sie die Modellierung fehleranf�lliger. Andererseits ist es schwerer,
die modellierten Daten zu erfassen und zu verstehen. Oftmals werden deshalb f�r verschiedene
Tests dieselben Daten verwendet, damit der Tester nur einen Satz von Testdaten verstehen
muss.

\todo{JW fragen: STU kurz einleiten?}


\section{Zielsetzung}

Die Modellierung von Testdaten soll vereinfacht werden. Dazu soll eine Modellierungssprache
entwickelt werden, die sowohl leicht verst�ndlich als auch leicht zu erlernen ist. Besonderer
Fokus liegt auf der Modellierung von Beziehungen. Die L�sung soll einfach zu benutzen sein
und sich in Entwicklungsumgebungen wie Eclipse integrieren lassen.

Dar�ber hinaus sollen sich Testdaten automatisch generieren lassen. Auch hier soll besondere
Aufmerksamkeit auf die Generierung von Beziehungen zwischen Datens�tzen gelegt werden. 

\section{Aufbau der Arbeit}

% 2 Grundlegende Konzepte
Zun�chst werden in \ref{chap:grundlagen} einige grundlegende Konzepte bzw. Technologien und
in der Arbeit verwendete Konventionen beschrieben, die f�r das weitere Verst�ndnis notwendig
sind.

- \ref{chap:anforderungen} Anforderungsanalyse / Fragestellung

- \ref{chap:modellierung} Entwurf einer Modellierungssprache f�r Testdaten

- \ref{chap:realiserungdsl} Realisierung der Sprache

- \ref{chap:generieren} Generieren von Testdaten

- \ref{chap:zusammenfassung} Zusammenfassung und Ausblick

\todo{Weitere Quellen und Zitate einbauen}
\section*{Ungenutzte Quellen}
\begin{enumerate}
	\item \cite[20ff]{DER_INTEGRATIONSTEST}
	\item \cite{MODELLGETRIEBENE_SOFTWAREENTWICKLUNG}
	\item \cite{DOMAIN_DRIVEN_DESIGN}
\end{enumerate}
